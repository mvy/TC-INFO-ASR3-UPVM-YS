%%%%%%%%%%%%%%%%%%%%%%%%%%%%%%%%%%%%%%%%%%%%%%%
%% Introduction aux Systèmes d'exploitation  %%
%%   * Historique                            %%
%%   * Principes fondamentaux                %%
%%   * Grandes classes de systèmes           %%
%%%%%%%%%%%%%%%%%%%%%%%%%%%%%%%%%%%%%%%%%%%%%%%

\title{Systèmes d'exploitation}
\subtitle{Les expressions regulières}

\author{Yves \textsc{Stadler}}\institute{Codasystem, UPV-M}

\date{\today}

\begin{document}


%%
% Page de Titre
%%
\begin{frame}
\titlepage
\end{frame}

\def\sectitle{Introduction}
\def\subsectitle{Vocabulaire}

\begin{frame}{\sectitle}
\begin{block}{\subsectitle}
Expression régulière / Expression rationnelle
\end{block}

\begin{block}{\subsectitle}
Parfois motif, ou en anglais pattern
\end{block}

\end{frame}



\def\subsectitle{Késako?}
\begin{frame}{\sectitle}
\begin{block}{\subsectitle}
\begin{itemize}
\item Chaîne de caractères
\item Décrit un ensemble de chaînes de caractères
\end{itemize}
\end{block}


\def\subsectitle{D'où est-ce que cela vient?}
\begin{block}{\subsectitle}
\begin{itemize}
\item De la théorie des mathématiques sur les langages
\item Ken Thompson les intègre dans qed -> ed -> grep
\end{itemize}
\end{block}
\end{frame}



\def\subsectitle{À quoi ça sert?}
\begin{frame}{\sectitle}
\begin{block}{\subsectitle}
\begin{itemize}
\item Décrire et classifier les langages
\item Editer, contrôler, manipuler du texte
\end{itemize}
\end{block}

\def\subsectitle{Comment les utiliser}
\begin{block}{\subsectitle}
\begin{itemize}
\item Lex
\item Expr
\item awk
\item Perl, TCL, Python
\end{itemize}
\end{block}
\end{frame}

\def\sectitle{Principes}
\def\subsectitle{Les bases}
\begin{frame}{\sectitle}
\begin{block}{\subsectitle}
\begin{itemize}
\item On veut décrire un ensemble de chaines par un motif
\item On va utiliser des alternatives et des quantifieurs
\item On rajoutera quelques méta-caractères
\end{itemize}
\end{block}
\end{frame}

\def\sectitle{Alternatives}
\def\subsectitle{Usage}
\begin{frame}[containsverbatim]{\sectitle}
\begin{block}{\subsectitle}
\begin{itemize}
\item La barre verticale | est l'opérateur d'alternation (opérateur d'alternative)
\item \verb!a|e! décrit les chaînes suivantes : "a" et "e"
\item \verb!ab|eb! : "ab" et "eb"
\item \verb!Bonjour|Aurevoir! : "Bonjour" et "Aurevoir"
\end{itemize}
\end{block}

\begin{block}{\subsectitle}
\begin{itemize}
\item Les parenthèses permettent de gérer la priorité et les groupements
\item Exemple avec "ex-équo", "ex-equo", "ex-aequo" et "ex-æquo" 
\item \verb!(ae|e)quo! et \verb!aequo|equo! désignent tout les deux les chaînes "aequo" et "equo"
\end{itemize}
\end{block}
\end{frame}



\def\ftitle{Quantifieurs de bases}
\def\blocktitle{0, 1 et plus}
\begin{frame}[containsverbatim]{\ftitle}
\begin{block}{\blocktitle}
\begin{itemize}
\item Un quantifieur quantifie le groupement le précédent
\item \verb!?! : le groupe peut apparaître zéro ou une fois
\item \verb!*! : le groupe peut apparaître zéro, une ou plusieurs fois
\item \verb!+! : le groupe doit apparaître au moins une fois, ou plus. 
\end{itemize}
\end{block}


\def\blocktitle{Exemple}
\begin{block}{\blocktitle}
\begin{itemize}
\item \verb!toto?! désigne : "tot" et "toto"
\item \verb!toto*! désigne : "tot", "toto", "totoo", "toto...o"
\item \verb!toto+! désigne : "toto", "totoo", "toto....o" mais pas "tot"
\item \verb!ex-(a?e|æ|é)quo! désigne : "ex-équo", "ex-equo", "ex-aequo" et "ex-æquo"
\end{itemize}
\end{block}

\end{frame}

\begin{frame}[containsverbatim]{\ftitle}
\begin{block}{\blocktitle}
\begin{itemize}
\item \verb!(a|b)*aaa! désigne toute chaîne commençant par une suite (nulle ou non) de caractères "a" ou "b" se terminant par "aaa"
\item \verb!(0|1|2|3|4|5|6|7|8|9)+! désigne tous les nombres entier (mêmes ceux commençant par des 0)
\end{itemize}
\end{block}

\def\blocktitle{Attention}
\begin{alertblock}{\blocktitle}
\begin{itemize}
\item Une chaîne vide peut être décrite par un motif
\item \verb!a?b?c?d?e?f?g?h?i?j?k?! décrit la chaîne vide, ou une suite de lettres de a à k se suivant dans l'ordre.
\item La chaîne vide correspond (match) aussi le motif \verb!(item)*!.
\item En revanche, elle le match pas le motif \verb!item*!, ni \verb!(item)+!.
\end{itemize}
\end{alertblock}
\end{frame}

\def\ftitle{Méta-caractères}
\begin{frame}[containsverbatim]{\ftitle}
\def\blocktitle{Classes}
\begin{block}{\blocktitle}
\begin{itemize}
\item \verb![...]! : N'importe quel caractères dans les crochets
\item \verb![^...]! : N'importe quel caractères qui n'est pas dans les crochets
\end{itemize}
\end{block}

\def\blocktitle{Intervalle}
\begin{block}{\blocktitle}
\begin{itemize}
\item \verb![a-z]! : toutes les lettres minuscules
\item \verb![A-Z]! : toutes les lettres majuscules
\item \verb![A-Za-z]! : toutes les lettres majuscules et miniscules
\item \verb![0-9]! : tous les chiffres.
\end{itemize}
\end{block}

\def\blocktitle{Exemples}
\begin{block}{\blocktitle}
\begin{itemize}
\item \verb!h[aeiouy]! désigne : "ha", "hi", "ho", "hu", "hy"
\item \verb!h[^aeiouy]! désigne : "hb", "hc", "hd", "hf" ... sauf "ha", "hi", "ho", "hu", "hy"
\end{itemize}
\end{block}
\end{frame}


\begin{frame}[containsverbatim]{\ftitle}
\def\blocktitle{Début, fin... joker!}
\begin{block}{\blocktitle}
\begin{itemize}
\item \verb!^! : début de phrase
\item \verb!$! : fin de phrase %$
\item \verb!.! : n'importe quel caractère
\end{itemize}
\end{block}

\def\blocktitle{Exemples}
\begin{block}{\blocktitle}
\begin{itemize}
\item \verb!^cours.*! : chaîne débutant par cours, et suivie d'autres caractères
\item \verb!^cours! : chaîne commençant par cours
\item \verb!cours$! : chaînes se terminant par cours %$
\end{itemize}
\end{block}
\end{frame}


\def\ftitle{POSIX ou PCRE}
\begin{frame}[containsverbatim]{\ftitle}
\def\blocktitle{Vocabulaire}
\begin{block}{\blocktitle}
\begin{itemize}
\item Beaucoup de syntaxe se sont développées pour les expressions régulières
\item Les deux principales sont celles préconisée par POSIX (Portable Open Software Interface X) et les PCRE (Perl Compatible Regular Expression)
\end{itemize}
\end{block}

\def\blocktitle{Notation}
\begin{block}{\blocktitle}
\begin{itemize}
\item On commence et termine les expressions PCRE par \verb!#!
\end{itemize}
\end{block}
\end{frame}

\def\ftitle{POSIX}
\begin{frame}[containsverbatim]{\ftitle}
\def\blocktitle{Outils}
\begin{block}{\blocktitle}
\begin{itemize}
\item ed
\item grep
\item sed
\item vi
\end{itemize}
\end{block}


%\item \verb!\{m,n\}	: m = minimum, n = maximum. (Si pas de premier argument : 0)
%\item On doit échapper les parenthèses : \verb!\( \)!

\def\blocktitle{Quantifieurs supplémentaires}
\begin{block}{\blocktitle}
\begin{itemize}
\item \verb!{m,n}!	: m = minimum, n = maximum. (Si pas de premier argument : 0, si le second est omis : infini)
\item Parenthèses : \verb!( )!
\item \verb!{0,1}! ou \verb!{,1}!: ?
\item \verb!{0,}! : *
\item \verb!{1,}! : +
\end{itemize}
\end{block}
\end{frame}

\begin{frame}[containsverbatim]{\ftitle}

\def\blocktitle{Attention}
\begin{alertblock}{\blocktitle}
\begin{itemize}
\item Dans vim ou emacs, les regexp étendue ne sont pas POSIX, il faut echapper les ( )  et \{ \}
\end{itemize}
\end{alertblock}

\def\blocktitle{Échappement}
\begin{block}{\blocktitle}
\begin{itemize}
\item \verb!(, ), [, ], ., *, ?, +, ^, |, $ et \! doivent être échappés pour avoir leur signification textuelle. %$
\item On utilise le backslash.
\end{itemize}
\end{block}
\end{frame}



\def\ftitle{Classes supplémentaires}
\begin{frame}[containsverbatim]{\ftitle}
\def\blocktitle{Classes POSIX}
\begin{block}{\blocktitle}
\begin{itemize}
\item \verb![:cntrl:]! 	Caractère de contrôle 	\verb![\x00-\x1F\x7F]!
\item \verb![:space:]! 	Espace blanc ou séparateur de ligne ou de paragraphe 	\verb![ \t\r\n\v\f]!
\item \verb![:blank:]! 	Espace blanc ou tabulation non séparateur de ligne ou de paragraphe 	\verb![ \t]!
\item \verb![:print:]! 	Espace simple ou caractère graphique visible 	\verb![\x20-\x7E]!
\item \verb![:graph:]! 	Caractère graphique visible 	\verb![\x21-\x7E]!
\item \verb![:punct:]! 	Caractère de ponctuation 	\verb<[!"#$%&'()*+,-./:;?@[\]_`{|}~]<
\end{itemize}
\end{block}
\end{frame}

\def\ftitle{Introduction}
\begin{frame}[containsverbatim]{\ftitle}
\def\blocktitle{Vocabulaire}
\begin{block}{\blocktitle}
\begin{itemize}
\item \verb![:alnum:]! 	Caractère alphanumérique 	\verb![0-9a-zA-Z]!
\item \verb![:digit:]! 	Chiffre décimal 	\verb![0-9]!
\item \verb![:xdigit:]! 	Chiffre hexadécimal 	\verb![0-9a-fA-F]!
\item \verb![:alpha:]! 	Caractère alphabétique 	\verb![a-zA-Z]!
\item \verb![:lower:]! 	Lettre minuscule 	\verb![a-z]!
\item \verb![:upper:]! 	Lettre capitale 	\verb![A-Z]!
\end{itemize}
\end{block}
\end{frame}

\def\ftitle{PCRE}
\begin{frame}[containsverbatim]{\ftitle}
\def\blocktitle{Quantificateurs}
\begin{block}{\blocktitle}
\begin{itemize}
\item Comme pour POSIX étendu
\end{itemize}
\end{block}

\def\blocktitle{Classes}
\begin{block}{\blocktitle}
\begin{itemize}
\item Dans une classes les caractères ont leur signification littérale.
\item Sauf \verb!# ] -!
\end{itemize}
\end{block}
\end{frame}


\begin{frame}[containsverbatim]{\ftitle}
\def\blocktitle{Classes abrégées}
\begin{block}{\blocktitle}
\begin{itemize}
\item \verb!\d! 	Chiffre : \verb![0-9]!
\item \verb!\D! 	Pas un chiffre : \verb![^0-9]!
\item \verb!\w! 	Mot : \verb![a-zA-Z0-9_]!
\item \verb!\W! 	Pas un mot : \verb![^a-zA-Z0-9_]!
\item \verb!\t! 	Tabulation
\item \verb!\n! 	Nouvelle ligne
\item \verb!\r! 	Retour chariot
\item \verb!\s! 	Espace blanc (correspond à \verb!\t \n \r!)
\item \verb!\S! 	Pas un espace blanc (\verb!\t \n \r!)
\end{itemize}
\end{block}
\end{frame}


\def\ftitle{Pourquoi l'un où l'autre}
\begin{frame}[containsverbatim]{\ftitle}
\def\blocktitle{Comparé à POSIX, PCRE permet}
\begin{block}{\blocktitle}
\begin{itemize}
\item     Plus rapide
\item     Utilisation des références arrières
\item     Capture de toutes les occurences
\item     Rendre un quantificateur non gourmand
\item     Utilisation des parenthèses non capturantes
\item     Utilisation d'une foule d'options
\item     Utilisation d'assertions, masques conditionnels
\item     Fonctions de callback
\end{itemize}
\end{block}

\def\blocktitle{Mais ....}
\begin{block}{\blocktitle}
Nous verrons cela la semaine prochaine! (Ouf!)
\end{block}

\end{frame}


\def\ftitle{That's all folks!}
\begin{frame}[containsverbatim]{\ftitle}
\def\blocktitle{Le mot de la fin}
\begin{block}{\blocktitle}
MOT
\end{block}
\end{frame}

\end{document}
