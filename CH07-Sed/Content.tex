%%%%%%%%%%%%%%%%%%%%%%%%%%%%%%%%%%%%%%%%%%%%%%%
%% Introduction aux Systèmes d'exploitation  %%
%%   * Historique                            %%
%%   * Principes fondamentaux                %%
%%   * Grandes classes de systèmes           %%
%%%%%%%%%%%%%%%%%%%%%%%%%%%%%%%%%%%%%%%%%%%%%%%

\title{Systèmes d'exploitation}
\subtitle{Stream EDitor}

\author{Yves \textsc{Stadler}}\institute{Codasystem, UPV-M}

\date{\today}

\begin{document}


%%
% Page de Titre
%%
\begin{frame}
\titlepage
\end{frame}


%!!!!!!!!!!!!!!!!!!!!!!!!!!!!!!!!!!!!!!!!!
\def\ftitle{Qu'est-ce que SED}
\begin{frame}[containsverbatim]{\ftitle}
%_________________________________________
\def\blocktitle{SED}
\begin{block}{\blocktitle}
\begin{itemize}
\item SED = Stream EDitor
\item Utilitaire Unix
\item Manipulation ligne par ligne du texte
\item Utilise des expressions regulières
\end{itemize}
\end{block}
%_________________________________________
\def\blocktitle{Exemple typique}
\begin{block}{\blocktitle}
\begin{itemize}
\item Remplacement : \verb!sed -e 's/Ancien/Nouveau/g' ne > ns!
\item "Substitue 'Nouveau' à 'Ancien' en utilisant ne comme input, et ns en output"
\end{itemize}
\end{block}
\end{frame}


%!!!!!!!!!!!!!!!!!!!!!!!!!!!!!!!!!!!!!!!!!
\def\ftitle{Histoire}
\begin{frame}[containsverbatim]{\ftitle}
%_________________________________________
\def\blocktitle{Ed}
\begin{block}{\blocktitle}
\begin{itemize}
\item Basé sur l'éditeur ed, toujours présent actuellement
\item À la base, ed se voit adjoindre une fonction pour travailler son flux d'entrée
\item "g/re/p" (global/regular expression/print)
\item Donnera grep
\item Puis sed
\end{itemize}
\end{block}
\end{frame}


%!!!!!!!!!!!!!!!!!!!!!!!!!!!!!!!!!!!!!!!!!
\def\ftitle{Syntaxe}
\begin{frame}[containsverbatim]{\ftitle}
%_________________________________________
\def\blocktitle{Ligne de commande}
\begin{block}{\blocktitle}
\begin{itemize}
\item \verb!sed -e 'actions' input!
\item On préfère souvent rediriger la sortie
\end{itemize}
\end{block}
%_________________________________________
\def\blocktitle{Actions}
\begin{block}{\blocktitle}
\begin{itemize}
\item Remplacement
\item Effacement
\item Actions conditionnelles
\end{itemize}
\end{block}
\end{frame}


%!!!!!!!!!!!!!!!!!!!!!!!!!!!!!!!!!!!!!!!!!
\begin{frame}[containsverbatim]{\ftitle}
%_________________________________________
\def\blocktitle{Syntaxe complète}
\begin{block}{\blocktitle}\begin{verbatim}
sed [-options] [commande] [<fichier(s)>]
sed [-n [-e commande] [-f script] [-i[.extension]] 
[l [cesure]] rsu] [<commande>] [<fichier(s)>]
\end{verbatim}
\end{block}
%_________________________________________
\def\blocktitle{Commandes}
\begin{block}{\blocktitle}
\begin{itemize}
\item \verb/[addr[,addr]][!]cmd[args]/
\item \begin{verbatim}[addr[,addr]]{
cmd1
cmd2
cmd3
\end{verbatim}
\item \verb![addr[,addr]]{cmd1;cmd2;cmd3}!
\end{itemize}
\end{block}
\end{frame}


%!!!!!!!!!!!!!!!!!!!!!!!!!!!!!!!!!!!!!!!!!
\begin{frame}[containsverbatim]{\ftitle}
%_________________________________________
\def\blocktitle{Affichage/Non affichage}
\begin{block}{\blocktitle}
\begin{itemize}
\item L'option \verb!-n ! force sed a ne rien afficher sur stdout
\item On va pouvoir utiliser p pour forcer l'affichage dans des commandes sed
\end{itemize}
\end{block}
%_________________________________________
\def\blocktitle{Sélection de ligne}
\begin{block}{\blocktitle}
\begin{itemize}
\item Afficher la troisième ligne : \verb!sed -n 3p fichier!
\item Afficher un intervalle de ligne (de 7 à 10) : \verb!sed -n '7,3 p' fichier!
\item Afficher un groupe de ligne (toutes les 2 lignes à partir de la 3): \verb!sed -n 3~2p fichier!
\item Afficher la dernière ligne d'un fichier : \verb!sed -n '$ p' fichier! %$
\end{itemize}
\end{block}
\end{frame}


%!!!!!!!!!!!!!!!!!!!!!!!!!!!!!!!!!!!!!!!!!
\begin{frame}[containsverbatim]{\ftitle}
%_________________________________________
\def\blocktitle{Ma regexp me manque!!}
\begin{block}{\blocktitle}
\begin{itemize}
\item Sélection (et affichage) de ligne par motif
\item \verb!sed -n '/motif/'! affiche toutes les lignes ou apparait le motif.
\item \verb!\!\verb!#exp#! change le délimteur pour \#
\item \verb!/begin/,/end/ p ! : du motif begin au motif end.
\item \verb!num,/motif/! \verb!/motif/,num! : de la ligne num au motif (et inverse).
\end{itemize}
\end{block}
\end{frame}

%!!!!!!!!!!!!!!!!!!!!!!!!!!!!!!!!!!!!!!!!!
\def\ftitle{Options}
\begin{frame}[containsverbatim]{\ftitle}
%_________________________________________
\def\blocktitle{Paramètres}
\begin{block}{\blocktitle}
\begin{itemize}
\item \verb!-e cmd! exécution d'une commande (peut-etre utilisé plusieurs fois
\item \verb!-n! silencieux
\item \verb!-i[Suffixe]! sauvegarde le fichier dans fichier.Suffixe et écrase le precédent
\item \verb!-r! Utilisation de RE étendue.
\end{itemize}
\end{block}
\end{frame}


%!!!!!!!!!!!!!!!!!!!!!!!!!!!!!!!!!!!!!!!!!
\def\ftitle{Fichiers de scripts}
\begin{frame}[containsverbatim]{\ftitle}
%_________________________________________
\def\blocktitle{Commentaires}
\begin{block}{\blocktitle}
\begin{itemize}
\item \verb!#Commentaire!
\item \verb!#n! au début : force l'option -n
\end{itemize}
\end{block}


%_________________________________________
\def\blocktitle{Commandes}
\begin{block}{\blocktitle}
\begin{itemize}
\item q : quitter
\item d : delete
\item p : print
\item n : next line
\item Accolades : groupement de commandes
\item s : substitution
\end{itemize}
\end{block}
\end{frame}


%!!!!!!!!!!!!!!!!!!!!!!!!!!!!!!!!!!!!!!!!!
\begin{frame}[containsverbatim]{\ftitle}
%_________________________________________
\def\blocktitle{Exemples}
\begin{block}{\blocktitle}
\begin{itemize}
\item Supprimer les lignes contenant "Vol pour Madrid"
\item /Vol pour Madrid/d
\item Afficher la ligne suivant la 4e.
\item 4 {n;p}
\item Afficher la quatrième ligne et quitter
\item 4 {n;q}
\end{itemize}
\end{block}
\end{frame}

%!!!!!!!!!!!!!!!!!!!!!!!!!!!!!!!!!!!!!!!!!
\def\ftitle{La substitution}
\begin{frame}[containsverbatim]{\ftitle}
%_________________________________________
\def\blocktitle{Substitution simple}
\begin{block}{\blocktitle}
\begin{itemize}
\item s/motif/motif2/
\item motif est une 'basic regular expression' (backslash de |)
\item On peut mettre des backreference dans motif2 (backslash parenthèse)
\item \& représente le motif en correspondance.
\end{itemize}
\end{block}
%_________________________________________
\def\blocktitle{Substitution composée}
\begin{block}{\blocktitle}
\begin{itemize}
\item s/motif/motif2/drapeaux
\end{itemize}
\end{block}
\end{frame}



%!!!!!!!!!!!!!!!!!!!!!!!!!!!!!!!!!!!!!!!!!
\begin{frame}[containsverbatim]{\ftitle}
%_________________________________________
\def\blocktitle{Les drapeaux}
\begin{block}{\blocktitle}
\begin{itemize}
\item g : global (toutes les occurences sur la ligne)
\item N (nombre) : la Nième occurence
\item p : print (affiche la ligne remplacée)
\item w : write (enregistre la ligne dans un fichier)
\item e : evaluate (fait interpréter une commande par le shell)
\item I : case insensitive
\item M : compliqué (pour le traitement de plusieurs lignes)
\end{itemize}
\end{block}
\end{frame}



%!!!!!!!!!!!!!!!!!!!!!!!!!!!!!!!!!!!!!!!!!
\begin{frame}[containsverbatim]{\ftitle}
%_________________________________________
\def\blocktitle{Exemples}
\begin{block}{\blocktitle}
\begin{itemize}
\item Substituer les lignes contenant "Vol pour Rio" par Annulé
\item s/Vol pour Rio/Annulé/
\item Ecrire l'heure à la place du nom du vol
\item s/Vol pour Rio/date/e
\item Trouver toutes les instances de Chapitre, quelque soit sa casse, et remplacer par CH
\item s/Chapitre/CH/gpi
\end{itemize}
\end{block}
\end{frame}

%!!!!!!!!!!!!!!!!!!!!!!!!!!!!!!!!!!!!!!!!!
\def\ftitle{Substitution}
\begin{frame}[containsverbatim]{\ftitle}
%_________________________________________
\def\blocktitle{y!}
\begin{block}{\blocktitle}
\begin{itemize}
\item \verb!y/AEIOU/aeiou/!
\item Remplace un caractère de la première chaine, par celui correspondant dans la seconde.
\end{itemize}
\end{block}
%_________________________________________
\def\blocktitle{Autres commandes}
\begin{block}{\blocktitle}
\begin{itemize}
\item a\\text : ajout de texte (après la mise en correspondance)
\item i\\text : ajout de texte (avant la mise en correspondance)
\item c\\text : échange la/les lignes(s) en correspondance
\item r fichier : lit un fichier
\item w fichier : écrit la ligne dans le fichier.
\item = : numéro de ligne
\end{itemize}
\end{block}
\end{frame}


%!!!!!!!!!!!!!!!!!!!!!!!!!!!!!!!!!!!!!!!!!
\begin{frame}[containsverbatim]{\ftitle}
%_________________________________________
\def\blocktitle{Exemples}
\begin{block}{\blocktitle}
\begin{itemize}
\item Ajouter Vol annulé a la suite du nom de vol
\item /Vol pour Rio/a Vol annulé
\item Après la 5e ligne
\item 5a Vol Annulée 
\end{itemize}
\end{block}
\end{frame}

\end{document}
