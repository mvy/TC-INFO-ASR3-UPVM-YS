%%%%%%%%%%%%%%%%%%%%%%%%%%%%%%%%%%%%%%%%%%%%%%%
%% Introduction aux Systèmes d'exploitation  %%
%%   * Historique                            %%
%%   * Principes fondamentaux                %%
%%   * Grandes classes de systèmes           %%
%%%%%%%%%%%%%%%%%%%%%%%%%%%%%%%%%%%%%%%%%%%%%%%

\title{Systèmes d'exploitation}
\subtitle{RegExp: Higher and Deeper}

\author{Yves \textsc{Stadler}}\institute{Université de Lorraine}

\date{\today}

\begin{document}


%%
% Page de Titre
%%
\begin{frame}
\titlepage
\end{frame}

\def\ftitle{Objectifs du chapitre}
\begin{frame}[containsverbatim]{\ftitle}
\def\blocktitle{Objectifs}
\begin{block}{\blocktitle}
\begin{itemize}
\item Savoir quelles commandes peuvent-être utilisées avec des regexp.
\item Savoir utiliser la capture et la \textit{backreference}
\item Contrôler sa gourmandise
\item Rechercher toutes les occurrences
\end{itemize}
\end{block}
\end{frame}


%!!!!!!!!!!!!!!!!!!!!!!!!!!!!!!!!!!!!!!!!!
\def\ftitle{Commandes}
\begin{frame}[containsverbatim]{\ftitle}
%_________________________________________
\def\blocktitle{POSIX}
\begin{block}{\blocktitle}
\begin{itemize}
\item ereg (macthing)
\item eregi (matching / case insensitive)
\item ereg\_replace (replacement)
\item eregi\_replace (case insensitive replacement)
\item split (découpe)
\item spliti (découpe insensible a la casse)
\end{itemize}
\end{block}
\end{frame}


%!!!!!!!!!!!!!!!!!!!!!!!!!!!!!!!!!!!!!!!!!
\begin{frame}[containsverbatim]{\ftitle}
%_________________________________________
\def\blocktitle{PCRE}
\begin{block}{\blocktitle}
\begin{itemize}
\item preg\_grep — Retourne un tableau avec les résultats de la recherche
\item preg\_match — Une seule occurence
\item preg\_match\_all — Toutes occurences
\item preg\_replace — rechercher et remplacer
\item preg\_replace\_callback — Recherche, remplace en utilisant une fonction
\item preg\_split — Eclatement de chaine
\end{itemize}
\end{block}
\end{frame}

%!!!!!!!!!!!!!!!!!!!!!!!!!!!!!!!!!!!!!!!!!
\def\ftitle{Capture}
\begin{frame}[containsverbatim]{\ftitle}
%_________________________________________
\def\blocktitle{La parenthèse}
\begin{block}{\blocktitle}
\begin{itemize}
\item Utiliser une parenthèse créér une pseudo-variable \verb!\i! qui stocke la valeur du sous-motif. %$
\end{itemize}
\end{block}
%_________________________________________
\def\blocktitle{Exemple}
\begin{block}{\blocktitle}
\begin{itemize}
\item \verb![0-9]{2}-[0-9]{2}-[0-9]{4}! désigne une chaine de type date JJ-MM-AAAA
\item \verb!([0-9]{2})-([0-9]{2})-([0-9]{4})! 
\item \verb!\1! prend la valeur de JJ
\item \verb!\2! prend la valeur de MM
\item \verb!\3! prend la valeur de AAAA %$
%\item En PCRE \verb!$i! devient \verb!\i! %$
%\item Utile pour le remplacement dans la partie motif de remplacement.
\end{itemize}
\end{block}
\end{frame}


%!!!!!!!!!!!!!!!!!!!!!!!!!!!!!!!!!!!!!!!!!
\begin{frame}[containsverbatim]{\ftitle}
\def\blocktitle{Exemple}
%_________________________________________
\begin{block}{\blocktitle}
\begin{itemize}
\item \verb!(.*)\1! représente les phrases en deux parties identiques ("pouetpouet", "haha", "toto")
\item \verb!t([aeiou])t\1! désigne toutes les chaînes tXtY ou X=Y= une voyelle
\item \verb!0[1-6]([0-9]{2}){2}\1\2! désigne les numéros de téléphones qui se terminent par deux groupes de 4 chiffres identiques
(06 08 12 08 12, 03 87 70 87 70, ...)
\end{itemize}
\end{block}
\end{frame}


%!!!!!!!!!!!!!!!!!!!!!!!!!!!!!!!!!!!!!!!!!
\def\ftitle{Non capturante, groupements nommés, backreference}
\begin{frame}[containsverbatim]{\ftitle}
%_________________________________________
\def\blocktitle{PCRE only!}
\begin{block}{\blocktitle}
\begin{itemize}
\item On peut annuler la capture avec les PCRE (et sans doute aussi dans POSIX étendu)
\item \verb!(?:...)!
\item Pour utiliser une backreference, il faut au moins le nombre de parenthèses requis.
\end{itemize}
\end{block}
\end{frame}

%%%%%%%%%%%%%%%%%%%%%%%%%%%%%%%%%%%%%%%%%%%%%%%%%%%%%%%%%%%%%%%%%%%%%%%%%%%%%%%%%%%%

%|||||||||||||||||||||||||||||||||||||||||
\def\ftitle{La gourmandise}
\begin{frame}[containsverbatim]{\ftitle}
%_________________________________________
\def\blocktitle{Analysons ceci :}
\begin{block}{\blocktitle}
\begin{itemize}
\item La RE : \verb!.*A.*!
\item Et la phrase : "zAzAzAz"
\item Comment va réagir la correspondance?
\end{itemize}
\end{block}
%_________________________________________
\def\blocktitle{Possibilités}
\begin{block}{\blocktitle}
\begin{itemize}
\item .* = z | A | .* = zAzAz
\item .* = zAz | A | .* = zAz
\item .* = zAzAz | A | .* = zAz
\end{itemize}
\end{block}
\end{frame}


%!!!!!!!!!!!!!!!!!!!!!!!!!!!!!!!!!!!!!!!!!
\begin{frame}[containsverbatim]{\ftitle}
%_________________________________________
\def\blocktitle{Est-un vilain défaut}
\begin{block}{\blocktitle}
\begin{itemize}
\item Dans les RE POSIX, les quantifieurs sont toujours gourmands (Plus c'est long... plus c'est bon)
\item Avec les PCRE, les quantifieurs ne sont gourmands que par défaut.
\end{itemize}
\end{block}

%_________________________________________
\def\blocktitle{Cure de désintox}
\begin{block}{\blocktitle}
\begin{itemize}
\item Pour supprimer la gourmandise d'un opérateur on utilise ?
\item \verb!.*?A.*!
\item Capture : .* = z | A | .* = zAzAz
\end{itemize}
\end{block}
\end{frame}


%!!!!!!!!!!!!!!!!!!!!!!!!!!!!!!!!!!!!!!!!!
\begin{frame}[containsverbatim]{\ftitle}
%_________________________________________
\def\blocktitle{Avec PHP}
\begin{block}{\blocktitle}
\begin{itemize}
\item Fonction ereg()
\item \verb!ereg('<a href=(.*)>', $text, $out);!
\item \begin{verbatim}$text="Une phrase avec <a href="http://www.codasystem
.com/mapage.php"> que je veux pouvoir matcher</a> mais pas 
le reste"! \end{verbatim}%$
\item out contiendra le résultat de la capture
\item POSIX: 'http://www.codasystem.com/mapage.php"> que je veux pouvoir matcher</a'
\end{itemize}
\end{block}

%_________________________________________
\def\blocktitle{PCRE}
\begin{block}{\blocktitle}
\begin{itemize}
\item Fonction preg\_match()
\item \verb!preg_match('<a href=(.*?)>', $text, $out);!
\item out contiendra le résultat de la capture
\item PCRE: "http://www.codasystem.com/mapage.php"
\end{itemize}
\end{block}
\end{frame}

%!!!!!!!!!!!!!!!!!!!!!!!!!!!!!!!!!!!!!!!!!
\def\ftitle{Assertions}

\begin{frame}[containsverbatim]{\ftitle}
%_________________________________________
\def\blocktitle{Assertions simples}
\begin{block}{\blocktitle}
\begin{itemize}
\item \verb!\b! Limite d'un mot
\item \verb!\B! pas limite d'un mot
\item \verb!^! début de phrase
\item \verb!$! fin de phrase %$
\end{itemize}
\end{block}
\end{frame}

\begin{frame}[containsverbatim]{\ftitle}
%_________________________________________
\def\blocktitle{Lookahead assertions}
\begin{block}{\blocktitle}
\begin{itemize}
\item (?=...), réussit si la regexp match l'expression à cette position.
\item (?!...), l'inverse
\item L'assertion laisse le curseur de lecture là où l'on commence a tester l'assertion (c'est à dire que si l'assertion est vraie on va relire depuis le début du groupe assertif)
\end{itemize}
\end{block}
%_________________________________________
\def\blocktitle{Exemple}
\begin{block}{\blocktitle}
\begin{itemize}
\item Matcher les fichiers et leurs extensions
\item \verb!.*[.].*$! %$
\item Mais sans les .bat
\end{itemize}
\end{block}
\end{frame}


%!!!!!!!!!!!!!!!!!!!!!!!!!!!!!!!!!!!!!!!!!
\def\ftitle{Solutions}
\begin{frame}[containsverbatim]{\ftitle}
%_________________________________________
\def\blocktitle{Ce qui ne marche pas}
\begin{block}{\blocktitle}
\begin{itemize}
\item \verb!.*[.][^b].*$!
\item \verb!.*[.]([^b]..|.[^a].|..[^t])$!
\item \verb!.*[.]([^b].?.?|.[^a]?.?|..?[^t]?)$! %$
\end{itemize}
\end{block}
%_________________________________________
\def\blocktitle{Solution avec lookahead}
\begin{block}{\blocktitle}
\begin{itemize}
\item \verb^.*[.](?!bat$).*$^
\item Si après le . il n'y a pas bat, on continue.
\item Le .* après l'assertion matchera l'extension (la lecture reste bloquée au début de l'assertion/)
\end{itemize}
\end{block}
\end{frame}


%!!!!!!!!!!!!!!!!!!!!!!!!!!!!!!!!!!!!!!!!!
\begin{frame}[containsverbatim]{\ftitle}
%_________________________________________
\def\blocktitle{Un pas en avant, trois pas en arrière!}
\begin{block}{\blocktitle}
\begin{itemize}
\item (?<=...), réussit si la regexp match l'expression en arrière de cette position.
\item (?<!...), réussit si ne match pas (en arrière)
\end{itemize}
\end{block}
\end{frame}

%!!!!!!!!!!!!!!!!!!!!!!!!!!!!!!!!!!!!!!!!!
\def\ftitle{Masques conditionnels}
\begin{frame}[containsverbatim]{\ftitle}
%_________________________________________
\def\blocktitle{Syntaxe}
\begin{block}{\blocktitle}
\begin{itemize}
\item \verb!(?(assertion ou backreference) motif si vrai | motif sur faux)!
\end{itemize}
\end{block}
%_________________________________________
\def\blocktitle{Exemple}
\begin{block}{\blocktitle}
\begin{itemize}
\item Si je détecte From dans un mail, je traite un motif d'adresse email
\item Si je détecte Subject, je traite une suite de mots
\end{itemize}
\end{block}
\end{frame}

%!!!!!!!!!!!!!!!!!!!!!!!!!!!!!!!!!!!!!!!!!
\def\ftitle{Failing attempts}
\begin{frame}[containsverbatim]{\ftitle}
%_________________________________________
\def\blocktitle{Capture vide, et groupe non participant}
\begin{block}{\blocktitle}
\begin{itemize}
\item \verb!(q?)b\1! macth b 
\item q? macth le vide, le groupe capture vide, b match b, \verb!1! doit matcher vide c'est ok.
\item \verb!(q)?b\1! ne match pas b
\item (q) match le vide (optionnel), le groupe ne capture rien, car il ne participe pas à l'expression. b match b, \verb!\1! va échouer car le groupe n'a pas participé.
\end{itemize}
\end{block}
\end{frame}


%!!!!!!!!!!!!!!!!!!!!!!!!!!!!!!!!!!!!!!!!!
\def\ftitle{Répetitions}
\begin{frame}[containsverbatim]{\ftitle}
%_________________________________________
\def\blocktitle{Que se passe-t-il pour la capture si}
\begin{block}{\blocktitle}
\begin{itemize}
\item J'utilise \verb!azerty([0-9])+!
\item La capture ([0-9]) risque d'intervenir plusieurs fois.
\item \verb!\1! ne se souviendra que de la dernière capture.
\end{itemize}
\end{block}
\end{frame}


%!!!!!!!!!!!!!!!!!!!!!!!!!!!!!!!!!!!!!!!!!
\def\ftitle{Forward reference}
\begin{frame}[containsverbatim]{\ftitle}
%_________________________________________
\def\blocktitle{Utilisons une capture avant de la définir}
\begin{block}{\blocktitle}
\begin{itemize}
\item Fonctionne avec PCRE
\item \verb!(\2deux|(un))+! va matcher : unundeux
\item Au début, \verb!\2! échoue a matcher u, le seconde alternative "un" réussi (un est capturé)
\item Deuxième itération \verb!\2! match "un" et deux match deux.
\end{itemize}
\end{block}
\end{frame}


%!!!!!!!!!!!!!!!!!!!!!!!!!!!!!!!!!!!!!!!!!
\def\ftitle{Quelques options}
\begin{frame}[containsverbatim]{\ftitle}
%_________________________________________
\def\blocktitle{PCRE}
\begin{block}{\blocktitle}
\begin{itemize}
\item On précise l'option après le \verb!#! final.
\item i : case insensitive
\item s : . macth aussi le retour à la ligne
\item U : ungreedy par défaut
\item x : permet d'écrire ses motifs sur plusieurs lignes, avec des commentaires (\verb!#!, il faut utiliser un autre délimiteur (ex. /))
\end{itemize}
\end{block}
\end{frame}
\end{document}
