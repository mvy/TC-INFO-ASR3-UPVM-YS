%%%%%%%%%%%%%%%%%%%%%%%%%%%%%%%%%%%%%%%%%%%%%%%
%% Introduction aux Systèmes d'exploitation  %%
%%   * Historique                            %%
%%   * Principes fondamentaux                %%
%%   * Grandes classes de systèmes           %%
%%%%%%%%%%%%%%%%%%%%%%%%%%%%%%%%%%%%%%%%%%%%%%%

\title{Systèmes d'exploitation}
\subtitle{AWK}

\author{Yves \textsc{Stadler}}\institute{Codasystem, UPV-M}

\date{\today}

\begin{document}


%%
% Page de Titre
%%
\begin{frame}
\titlepage
\end{frame}


%!!!!!!!!!!!!!!!!!!!!!!!!!!!!!!!!!!!!!!!!!
\def\ftitle{Historique}
\begin{frame}[containsverbatim]{\ftitle}
%_________________________________________
\def\blocktitle{Vocabulaire}
\begin{block}{\blocktitle}
\begin{itemize}
\item 1977, langage de programmation, UNIX
\item Alfred V. Aho, Peter J. Weinberger et Brian W. Kernighan.
\item Créer pour remplacer grep et sed.
\item Dispose d'une grande souplesse
\item Plus de possibilité "programmative" que sed.
\end{itemize}
\end{block}
\end{frame}


%!!!!!!!!!!!!!!!!!!!!!!!!!!!!!!!!!!!!!!!!!
\def\ftitle{Premiers pas}
\begin{frame}[containsverbatim]{\ftitle}
%_________________________________________
\def\blocktitle{Appel}
\begin{block}{\blocktitle}
\begin{itemize}
\item \verb|#!/usr/bin/awk -f|
\item \verb!awk -f fscript < finput!
\item \verb!ls | awk -f fscript!
\end{itemize}
\end{block}
\end{frame}

%!!!!!!!!!!!!!!!!!!!!!!!!!!!!!!!!!!!!!!!!!
\def\ftitle{Terminologie}
\begin{frame}[containsverbatim]{\ftitle}
%_________________________________________
\def\blocktitle{Enregistrement}
\begin{block}{\blocktitle}
\begin{verbatim}
Fichier := record \n record ... \n record \n EOF;
\end{verbatim}
\end{block}
%_________________________________________
\def\blocktitle{Champs}
\begin{block}{\blocktitle}
\begin{verbatim}
record := field SEP field ... SEP field SEP field \n;
\end{verbatim}
\end{block}
\end{frame}


%!!!!!!!!!!!!!!!!!!!!!!!!!!!!!!!!!!!!!!!!!
\def\ftitle{Terminologie}
\begin{frame}[containsverbatim]{\ftitle}
%_________________________________________
\def\blocktitle{Premières lignes}
\begin{block}{\blocktitle}
\begin{itemize}
\item Le dièse est commentaire
\item \verb!Condition { actions }!
\item ; ou \verb!\n! pour séparer les actions
\item \begin{verbatim}Condition {
action1
action2; action3
}
\end{verbatim}
\item \begin{verbatim}Condition \
{
actions
}
\end{verbatim}

\end{itemize}
\end{block}
\end{frame}


%!!!!!!!!!!!!!!!!!!!!!!!!!!!!!!!!!!!!!!!!!
\def\ftitle{Le début}
\begin{frame}[containsverbatim]{\ftitle}
%_________________________________________
\def\blocktitle{BEGIN}
\begin{block}{\blocktitle}
\begin{itemize}
\item Condition : \verb!BEGIN { actions }!
\item S'exécute avant le début de traitement du fichier
\item Pratique pour initialiser les variables, ou afficher un message de bienvenue
\item \verb!BEGIN { print "Debut du script AWK" }!
\end{itemize}
\end{block}
%_________________________________________
\def\blocktitle{END}
\begin{block}{\blocktitle}
\begin{itemize}
\item Condition : \verb!END { actions }!
\item S'exécute après le traitement du fichier
\item Pratique pour afficher les résultats, ou un message de fin
\item \verb!END { print "Fin du script AWK" }!
\end{itemize}
\end{block}
\end{frame}


%!!!!!!!!!!!!!!!!!!!!!!!!!!!!!!!!!!!!!!!!!
\def\ftitle{Variables}
\begin{frame}[containsverbatim]{\ftitle}
%_________________________________________
\def\blocktitle{Typage}
\begin{block}{\blocktitle}
\begin{itemize}
\item Comme bash, awk ne connait pas de type
\item \verb!variable_1 = "Bonjour"!
\item \verb!variable_1 = 1!
\item Le type est interprété en fonction du contexte
\end{itemize}
\end{block}

%_________________________________________
\def\blocktitle{Tableau}
\begin{block}{\blocktitle}
\begin{itemize}
\item Comme en bash : \verb!tab[0] = "Bonjour"!
\item \verb!tab["item1"] = 1!
\end{itemize}
\end{block}
\end{frame}


%!!!!!!!!!!!!!!!!!!!!!!!!!!!!!!!!!!!!!!!!!
\def\ftitle{Variables spéciales}
\begin{frame}[containsverbatim]{\ftitle}
%_________________________________________
\def\blocktitle{Billet vert}
\begin{block}{\blocktitle}
\begin{itemize}
\item \verb!$0! désigne toute la ligne
\item \verb!$i! désigne le ième champ de la ligne
\item Séparateur : espace, tabulation (Peut être modifier)
\item \verb! n = 2; print $n! affiche le champ 2%$
\end{itemize}
\end{block}
%_________________________________________
\def\blocktitle{Variables globales}
\begin{block}{\blocktitle}
\begin{itemize}
\item \verb!$NF! nombre de champ sur la ligne (number of fields)%$
\item \verb!$FS! contient l'expression pour les séparateurs de champs (peut être une RE)%$
\item \verb!$NR! nombre de ligne (number of records)%$
\item \verb!$ORS! séparateur de ligne (\verb!\n! par défaut)%$
\end{itemize}
\end{block}
\end{frame}


%!!!!!!!!!!!!!!!!!!!!!!!!!!!!!!!!!!!!!!!!!
\def\ftitle{Exemples}
\begin{frame}[containsverbatim]{\ftitle}
%_________________________________________
\def\blocktitle{Exemples}
\begin{block}{\blocktitle}
\begin{verbatim}
BEGIN { print "Le script qui affiche le premier champ" }
{ print $1 }
END { print "Le script est fini" }

BEGIN { print "Le script qui affiche le nombre de 
ligne sans utiliser $NR"
total = 0 }
{ total = total + 1 }
END { print "Le total de lignes est " $NR " ou " total }
\end{verbatim}%$
\end{block}
\end{frame}

%!!!!!!!!!!!!!!!!!!!!!!!!!!!!!!!!!!!!!!!!!
\def\ftitle{Conditions}
\begin{frame}[containsverbatim]{\ftitle}
%_________________________________________
\def\blocktitle{Opérateurs}
\begin{block}{\blocktitle}
\begin{itemize}
\item == != <= >= < >
\item \&\& ||
\item \verb!+ - / * % ^!
\item \verb!+= -= /= *= %/ ++ --!
\end{itemize}
\end{block}
%_________________________________________
\def\blocktitle{Motifs}
\begin{block}{\blocktitle}
\begin{itemize}
\item \verb!/regexp/ { actions }!
\item \verb!var ~ /regexep/ { actions }!
\item \verb|var !~ /regexp/ { actions }|
\end{itemize}
\end{block}
\end{frame}


%!!!!!!!!!!!!!!!!!!!!!!!!!!!!!!!!!!!!!!!!!
\def\ftitle{Exemples}
\begin{frame}[containsverbatim]{\ftitle}
%_________________________________________
\def\blocktitle{Script 1}
\begin{block}{\blocktitle}
\begin{verbatim}
BEGIN { print "Le script qui compte le nombre
 de ligne contenant des nombres"
total = 0 }
/[0-9]+/ { total = total + 1 }
END { print "Le total est " total }
\end{verbatim}

\end{block}
%_________________________________________
\def\blocktitle{Script 2}
\begin{block}{\blocktitle}
\begin{verbatim}
BEGIN { print "Le script qui compte le nombre
 de ligne dont le champ 1 est un nombres"
total = 0 }
$1 ~ /[0-9]+/ { total = total + 1 }
END { print "Le total est " total }
\end{verbatim}%$

\end{block}
\end{frame}

%!!!!!!!!!!!!!!!!!!!!!!!!!!!!!!!!!!!!!!!!!
\def\ftitle{Fonctions}
\begin{frame}[containsverbatim]{\ftitle}
%_________________________________________
\def\blocktitle{Arithmetiques}
\begin{block}{\blocktitle}
\begin{itemize}
\item sqrt(x) 	racine carrée
\item atan2(y,x)  	 arctangente de x/y en (radians)
\item cos(x) 	cosinus (radians)
\item sin(x) 	sinus (radians)
\item exp(x) 	exponentielle $e^{x}$
\item int(x) 	valeur entière
\item log(x) 	logarithme
\item rand() 	nombre aléatoire entre 0 et 1
\item srand(x) 	initialisation de rand
\end{itemize}
\end{block}

%_________________________________________
\def\blocktitle{Affichage}
\begin{block}{\blocktitle}
\begin{itemize}
\item print
\item printf()
\end{itemize}
\end{block}
\end{frame}

%!!!!!!!!!!!!!!!!!!!!!!!!!!!!!!!!!!!!!!!!!
\def\ftitle{Boucles, etc...}
\begin{frame}[containsverbatim]{\ftitle}
%_________________________________________
\def\blocktitle{Si le père noel exite alors ...}
\begin{block}{\blocktitle}
\begin{itemize}
\item \begin{verbatim}
if (condition)
	action
\end{verbatim}
\item \begin{verbatim}
if (condition) {
	actions
} [else {
	actions
}]
\end{verbatim}
\end{itemize}
\end{block}
%_________________________________________
\def\blocktitle{Pour...}
\begin{block}{\blocktitle}
\begin{itemize}
\item \begin{verbatim}
for (start; condition; step) {
	actions
} #C-like
\end{verbatim}
\item \begin{verbatim}
for (var in tableau) {
	actions
} #Bash-like
\end{verbatim}
\end{itemize}
\end{block}
\end{frame}


%!!!!!!!!!!!!!!!!!!!!!!!!!!!!!!!!!!!!!!!!!
\def\ftitle{Introduction}
\begin{frame}[containsverbatim]{\ftitle}
%_________________________________________
\def\blocktitle{Vocabulaire}
\begin{block}{\blocktitle}
\begin{verbatim}
 awk 'BEGIN { print "Mémorisation de votre fichier " FILENAME }
                 {memfile [NR] = $0 }
           END   { for ( i = NR ; i >= 1 ; i-- ) {
                    print i ":" memfile[i]
                    }
	          print "Fin"
                 } ' fichier 
\end{verbatim}%$
\end{block}
\end{frame}


%!!!!!!!!!!!!!!!!!!!!!!!!!!!!!!!!!!!!!!!!!
\def\ftitle{Tant qu'il y aura des ombres}
\begin{frame}[containsverbatim]{\ftitle}
%_________________________________________
\def\blocktitle{While}
\begin{block}{\blocktitle}
\begin{itemize}
\item \begin{verbatim}
while (condition)
{
	actions
}
\end{verbatim}
\item \begin{verbatim}
do
{
	actions
} while (condition)
\end{verbatim}
\end{itemize}
\end{block}
%_________________________________________
\def\blocktitle{Contrôle}
\begin{block}{\blocktitle}
\begin{itemize}
\item break;
\item continue;
\end{itemize}
\end{block}
\end{frame}

\end{document}
