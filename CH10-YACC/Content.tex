%%%%%%%%%%%%%%%%%%%%%%%%%%%%%%%%%%%%%%%%%%%%%%%
%% Introduction aux Systèmes d'exploitation  %%
%%   * Historique                            %%
%%   * Principes fondamentaux                %%
%%   * Grandes classes de systèmes           %%
%%%%%%%%%%%%%%%%%%%%%%%%%%%%%%%%%%%%%%%%%%%%%%%

\title{Systèmes d'exploitation}
\subtitle{Lex et Yacc : partie II (Yacc)}

\author{Yves \textsc{Stadler}}\institute{Codasystem, UPV-M}

\date{\today}

\begin{document}


%%
% Page de Titre
%%
\begin{frame}
\titlepage
\end{frame}


%!!!!!!!!!!!!!!!!!!!!!!!!!!!!!!!!!!!!!!!!!
\def\ftitle{Introduction}
\begin{frame}[containsverbatim]{\ftitle}
%_________________________________________
\includegraphics[width=\textwidth]{images/yak.png}
\end{frame}

%!!!!!!!!!!!!!!!!!!!!!!!!!!!!!!!!!!!!!!!!!
\def\ftitle{Introduction}
\begin{frame}[containsverbatim]{\ftitle}
%_________________________________________
\def\blocktitle{Yet Another Compiler Compiler}
\begin{block}{\blocktitle}
\begin{itemize}
\item Analyseur grammatical
\end{itemize}
\end{block}
%_________________________________________
\def\blocktitle{Historique}
\begin{block}{\blocktitle}
\begin{itemize}
\item Stephen C. Johnson at AT\&T 
\item Maintenant nommé bison
\end{itemize}
\end{block}
\end{frame}


%!!!!!!!!!!!!!!!!!!!!!!!!!!!!!!!!!!!!!!!!!
\def\ftitle{Introduction}
\begin{frame}[containsverbatim]{\ftitle}
%_________________________________________
\def\blocktitle{Principe}
\begin{block}{\blocktitle}
\begin{itemize}
\item Décrire des tokens (éléments de langages)
\item Écire une liste de règles qui décrivent un langage (Comment s'organisent les tokens)
\item Générer un programme qui va comprendre un langage
\item Une calculatrice comprends le langage des expressions mathématiques
\item Le fichier .yy donnera une source dans le langage choisi
\end{itemize}
\end{block}
\end{frame}


%!!!!!!!!!!!!!!!!!!!!!!!!!!!!!!!!!!!!!!!!!
\def\ftitle{Structure d'un fichier YACC}
\begin{frame}[containsverbatim]{\ftitle}
%_________________________________________
\def\blocktitle{Général}
\begin{block}{\blocktitle}
\begin{itemize}
\item Pareil qu'un fichier Lex
\item \begin{verbatim}
Definition section
%%
Rule section
%%
Additionnal code
\end{verbatim}
\item Le fichier aura généralement l'extension .yy
\end{itemize}
\end{block}
\end{frame}


%!!!!!!!!!!!!!!!!!!!!!!!!!!!!!!!!!!!!!!!!!

\begin{frame}[containsverbatim]{\ftitle}
%_________________________________________
\def\blocktitle{Definitions}
\begin{block}{\blocktitle}
\begin{itemize}
\item Includes comme en lex (\verb!%{ ... %}!)
\item Définition de tokens
\item Un token est un élément de syntaxe par exemple PLUS (\verb!%token PLUS!)
\item Un token ne réprésente rien! Il n'a aucune valeur! Il a juste un sens!
\item Exemple le token PLUS ne correspond pas à la chaîne "+"
\item Le token PLUS correspond au sens de l'addition (On expliquera son contexte d'utilisation dans la section suivante)
\item Un axiome (\verb!%axiom start!) est le nom de l'axiome de la grammaire. Par défaut c'est la première règle
\item Supposons que nous avons les tokens suivants : NOMBRE PLUS  MOINS FOIS  DIVISE  PUISSANCE

\end{itemize}
\end{block}
\end{frame}



%!!!!!!!!!!!!!!!!!!!!!!!!!!!!!!!!!!!!!!!!!

\begin{frame}[containsverbatim]{\ftitle}
%_________________________________________
\def\blocktitle{Productions}
\begin{block}{\blocktitle}
\begin{itemize}
\item Donner un sens à nos token
\item \begin{verbatim}
notion_grammaticale:
	  notionOuToken1 {action sémantique}
	| notionOuToken2 {action sémantique}
	...
	| notionOuToken3 {action sémantique}
	;
\end{verbatim}
\item NotionOuTokeni est soit une autre production/notion de cette section, soit un token de la section précédente. (Peut être composé)
\item Les actions sont effectuées lorsque YACC réussi à donner un sens à le notionOuToken qui précède.
\end{itemize}
\end{block}
\end{frame}


%!!!!!!!!!!!!!!!!!!!!!!!!!!!!!!!!!!!!!!!!!

\begin{frame}[containsverbatim]{\ftitle}
%_________________________________________
\def\blocktitle{Exemple : expressions mathématiques simples}
\begin{block}{\blocktitle}
\begin{verbatim}
EXPRESSION:   NOMBRE
			| EXPRESSION PLUS EXPRESSION {//calcul de l'addition}
			| EXPRESSION MOINS EXPRESION {//calcul de l'addition}
			| EXPRESSION FOIS EXPRESION {//calcul de l'addition}
			| EXPRESSION DIVISE EXPRESION {//calcul de l'addition}
			;
\end{verbatim}
\end{block}

%_________________________________________
\def\blocktitle{Lecture}
\begin{block}{\blocktitle}
\begin{itemize}
\item On dit qu'une expression est décrite par :
\item Une addition de deux autres expressions
\item Une soustraction de deux expressions
\item Une multiplication de deux expressions
\item Une division de deux expressions
\end{itemize}
\end{block}
\end{frame}



%!!!!!!!!!!!!!!!!!!!!!!!!!!!!!!!!!!!!!!!!!

\begin{frame}[containsverbatim]{\ftitle}
%_________________________________________
\def\blocktitle{Associativité}
\begin{block}{\blocktitle}
Dans notre cas EXPRESION PLUS EXPRESSION il y a ambiguité, YACC ne sait pas s'il doit chercher d'abord la première expression ou d'abord la seconde.
\begin{itemize}
\item On résoud le problème en précisant l'associativité de PLUS
\item \verb!%left PLUS! va dire que l'on commence par la gauche (par exemple pour une addition)
\item \verb!%right PUISSANCE! va dire que l'on commence par évaluer a right (par exemple pour une puissance)
\item On peut rendre un token non associatif : \verb!%nonassoc EGAL!
\end{itemize}
\end{block}

\end{frame}


%!!!!!!!!!!!!!!!!!!!!!!!!!!!!!!!!!!!!!!!!!
\begin{frame}[containsverbatim]{\ftitle}
%_________________________________________
\def\blocktitle{Précédence}
\begin{block}{\blocktitle}
YACC ne sait pas non plus dans une expression composée (ex: 2 + 3 * 5) s'il doit commencer d'abord par la règle PLUS ou la règle FOIS
\begin{itemize}
\item On indique grâce à \verb!%left et %right! la priorité des opérateurs
\item Première ligne moins prioritaire
\item Dernière ligne plus prioritaire
\begin{verbatim}
%left PLUS MOINS
%left FOIS DIVISE
\end{verbatim}
\end{itemize}
\end{block}
\end{frame}


%!!!!!!!!!!!!!!!!!!!!!!!!!!!!!!!!!!!!!!!!!
\begin{frame}[containsverbatim]{\ftitle}
%_________________________________________
\def\blocktitle{Code additionnel}
\begin{block}{\blocktitle}
\begin{itemize}
\item On peut comme en lex y définir certaines fonctions dont
\item \begin{verbatim}
int yyerror(char *s) {
  printf("%s\n",s);
}

int main(void) {
  yyparse();
}
\end{verbatim}
\end{itemize}
\end{block}
\end{frame}




%!!!!!!!!!!!!!!!!!!!!!!!!!!!!!!!!!!!!!!!!!
\def\ftitle{Lex et YACC}
\begin{frame}[containsverbatim]{\ftitle}
%_________________________________________
\def\blocktitle{Principe}
\begin{block}{\blocktitle}
\begin{itemize}
\item Utiliser la reconnaissance grammaticale pour donner un sens à l'analyse lexicale.
\item Exemple créer un programme C qui va pouvoir interpréter nos expressions mathématique et calculer le résultat
\item Variable commune : yylval de type YYSTYPE (on pourra faire un define YYSTYPE double par exemple.)
\end{itemize}
\end{block}
\end{frame}


%!!!!!!!!!!!!!!!!!!!!!!!!!!!!!!!!!!!!!!!!!
\begin{frame}[containsverbatim]{\ftitle}
%_________________________________________
\def\blocktitle{Exmple : fichier lex}
\begin{block}{\blocktitle}
\begin{verbatim}
SPACE    [ \t]+
DIGIT   [0-9]
INT    {DIGIT}+
FLOAT    {DIGIT}("."{DIGIT})?
%%
{SPACE}  { /* Ignorés */ }
{FLOAT}		{ yylval=atof(yytext);
      		 return(NOMBRE);
    		}
"+"   return(PLUS);
"-"   return(MOINS);
"*"   return(FOIS);
"/"   return(DIVISE);
"\n"  return(FIN);
\end{verbatim}
\end{block}
\end{frame}


%!!!!!!!!!!!!!!!!!!!!!!!!!!!!!!!!!!!!!!!!!
\begin{frame}[containsverbatim]{\ftitle}
%_________________________________________
\def\blocktitle{Exemple : fichier yacc}
\begin{block}{\blocktitle}
\begin{itemize}
\item On a déjà vu comment définir les tokens et leurs priorités
\item On choisit \verb!%axiom LIGNE!
\item On met aussi les règles que l'on a choisis
\item Comment faire maintenant pour calculer quelque chose?
\end{itemize}

%_________________________________________
\def\blocktitle{Variable}
\begin{block}{\blocktitle}
\begin{itemize}
\item notion: TOKEN1 TOKEN2 TOKEN3
\item \verb!$$! représente la valeur de la notion
\item \verb!$i! représente la valeur du token i %$
\item \verb!EXPRESSION : NOMBRE {$$=$1}! expression va prendre la valeur du token (calculer par la reconnaissance lex)
\item \verb!EXPRESSION : EXPRESSION PLUS EXPRESSION {$$=$1 + $2}! expression va prendre la valeur de la 1er plus la seconde expression (Calculés par une autre notion YACC)%$
\end{itemize}
\end{block}
\end{block}
\end{frame}


%!!!!!!!!!!!!!!!!!!!!!!!!!!!!!!!!!!!!!!!!!

\begin{frame}[containsverbatim]{\ftitle}
%_________________________________________
\def\blocktitle{Exemple}
\begin{block}{\blocktitle}
\begin{verbatim}
EXPRESSION:	  NOMBRE {$$=$1}
			| EXPRESSION PLUS EXPRESSION {$$=$1+$3}
			| EXPRESSION MOINS EXPRESSION {$$=$1-$3}
			| EXPRESSION DIVISE EXPRESSION {$$=$1/$3}
			| EXPRESSION FOIS EXPRESSION {$$=$1*$3}			
			;

LIGNE	:  LIGNE FIN {printf("Le resultat est %f\n", $1);}
		| FIN
		| error FIN {yyerrok; /* Rattrapage d'erreur */}			
\end{verbatim}
\end{block}
\end{frame}

\end{document}
