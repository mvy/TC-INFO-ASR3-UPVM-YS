%%%%%%%%%%%%%%%%%%%%%%%%%%%%%%%%%%%%%%%%%%%%%%%
%% Introduction aux Systèmes d'exploitation  %%
%%   * Historique                            %%
%%   * Principes fondamentaux                %%
%%   * Grandes classes de systèmes           %%
%%%%%%%%%%%%%%%%%%%%%%%%%%%%%%%%%%%%%%%%%%%%%%%

\title{Makefile}
\subtitle{Comment automatiser la chaine de compilation}

\author{Yves \textsc{Stadler}}\institute{Codasystem, UPV-M}

\date{\today}

\begin{document}


%%
% Page de Titre
%%
\begin{frame}
\titlepage
\end{frame}


%!!!!!!!!!!!!!!!!!!!!!!!!!!!!!!!!!!!!!!!!!
\def\ftitle{Introduction}
\begin{frame}[containsverbatim]{\ftitle}
%_________________________________________
\def\blocktitle{Historique et description}
\begin{block}{\blocktitle}
\begin{itemize}
\item Il existe de multiples utilitaires makefile.
\item Ce n'est pas normalisé
\item Nous parlerons de GNU make.
\end{itemize}
\end{block}


%_________________________________________
\def\blocktitle{Utilité}
\begin{block}{\blocktitle}
\begin{itemize}
\item Permet d'exécuter des commandes séquentiellement
\item A la différence des scripts, make n'exécute les commandes que si nécessaire
\item Make facilite la compilation et les éditions de liens
\end{itemize}
\end{block}
\end{frame}


%!!!!!!!!!!!!!!!!!!!!!!!!!!!!!!!!!!!!!!!!!
\def\ftitle{Introduction}
\begin{frame}[containsverbatim]{\ftitle}
%_________________________________________
\def\blocktitle{Historique}
\begin{block}{\blocktitle}
\begin{itemize}
\item Docteur Stuart Feldman, en 1977.
\item Bell Labs
\item Deux principaux, BSD et GNU
\item Maintenant pour les grands projets, on utilise des outils comme autoconf
\end{itemize}
\end{block}
\end{frame}


%!!!!!!!!!!!!!!!!!!!!!!!!!!!!!!!!!!!!!!!!!
\def\ftitle{Fonctionnement}
\begin{frame}[containsverbatim]{\ftitle}
%_________________________________________
\def\blocktitle{Règles}
\begin{block}{\blocktitle}
\begin{verbatim}
cible1: dependance1 dependance2 ...
  commande1
  commande2
  ...
  
cible2: ...
  ...
\end{verbatim}
\end{block}


%_________________________________________
\def\blocktitle{Projet}
\begin{block}{\blocktitle}
\begin{itemize}
\item Fichiers des commandes mathématiques
\item lmath.c lmath.h main.c
\end{itemize}
\end{block}
\end{frame}


%!!!!!!!!!!!!!!!!!!!!!!!!!!!!!!!!!!!!!!!!!
\begin{frame}[containsverbatim]{\ftitle}
%_________________________________________
\def\blocktitle{lmath.h}
\begin{block}{\blocktitle}
\begin{verbatim}
/* Fonction mathématique : addition */
int ajoute(int a, int b);
/* Fonction mathématique : multiplication */
int multiplie(int a, int b);
/* Fonction mathématique : soustraction */
int soustrait(int a, int b);
/* Fonction mathématique : division */
int divise(int a, int b);
\end{verbatim}
\end{block}
\end{frame}


%!!!!!!!!!!!!!!!!!!!!!!!!!!!!!!!!!!!!!!!!!
\def\ftitle{Introduction}
\begin{frame}[containsverbatim]{\ftitle}
%_________________________________________
\def\blocktitle{lmath.c}
\begin{block}{\blocktitle}
\begin{verbatim}
/* Corps de la fonction addition */
int addition(int a, int b) {
  return a + b;
}

/* Corps de la fonction multiplication */
int multiplie(int a, int b) {
  return a * b;
}

/* Corps de la fonction soustraction */
int soustrait(int a, int b) {
  return a - b;
}

/* Corps de la fonction division */
int divise(int a, int b) {
  return a / b;
}
\end{verbatim}
\end{block}
\end{frame}


%!!!!!!!!!!!!!!!!!!!!!!!!!!!!!!!!!!!!!!!!!

\begin{frame}[containsverbatim]{\ftitle}
%_________________________________________
\def\blocktitle{main.c}
\begin{block}{\blocktitle}
\begin{verbatim}
#include "lmath.h"

int main(void) {
  addition(1, 1);
  multiplication(addition(1,3), 5);
  division(9, 3);
}
\end{verbatim}
\end{block}
\end{frame}


%!!!!!!!!!!!!!!!!!!!!!!!!!!!!!!!!!!!!!!!!!
\def\ftitle{Première règles}
\begin{frame}[containsverbatim]{\ftitle}
%_________________________________________
\def\blocktitle{Chaîne}
\begin{block}{\blocktitle}
\begin{itemize}
\item Compiler lmath.c si on a modifier le corps d'une fonction en fichier .o (sans édition de lien)
\item Compiler main.c sans édition de lien si modifié
\item Faire l'édition de lien entre les deux fichiers objets
\begin{verbatim}
gcc -c -o lmath.o lmath.h
gcc -c -o main.o main.c
gcc -o executable main.o lmath.o
\end{verbatim}
\end{itemize}
\end{block}

%_________________________________________
\def\blocktitle{Makefile de base}
\begin{block}{\blocktitle}
main: lmath.o main.o
  gcc -o executable main.o lmath.o
  
lmath.o: lmath.c
  gcc -c -o lmath.o lmath.h
  
main.o: main.c lmath.h
  gcc -c -o main.o main.c
  
\end{block}
\end{frame}


%!!!!!!!!!!!!!!!!!!!!!!!!!!!!!!!!!!!!!!!!!
\begin{frame}[containsverbatim]{\ftitle}
%_________________________________________
\def\blocktitle{Explication}
\begin{block}{\blocktitle}
\begin{itemize}
\item La première règle est la règle par défaut
\item Pour construire la cible main (nom symbolique) nous avons besoin des fichiers lmath.o et main.o
\item Si on a ces deux fichiers, on peut exécuter la commande en dessous
\item Si il manque un de ces fichiers, on va chercher la règle correspondante
\item Si lmath.o n'existe pas, on a une règle nommée lmath.o
\item Si lmath.o existe, on vérifie quand même la règle, car lmath.o nécéssite lmath.c. Est-ce que lmath.c à été modifié?
\end{itemize}
\end{block}
\end{frame}


%!!!!!!!!!!!!!!!!!!!!!!!!!!!!!!!!!!!!!!!!!
\begin{frame}[containsverbatim]{\ftitle}
%_________________________________________
\def\blocktitle{Problèmes}
\begin{block}{\blocktitle}
\begin{itemize}
\item Les fichiers temporaires restent sur le disque
\item On ne peut pas forcer une génération complète du projet
\item On ne peut rien personnaliser
\end{itemize}
\end{block}


%_________________________________________
\def\blocktitle{Makefile avancé}
\begin{block}{\blocktitle}
\begin{itemize}
\item Règle dites "standards"
\item all : regroupe toutes les dépendances
\item clean : supprime les fichiers intermédiaires
\item mrproper : supprime tout ce qui peut-être regénéré
\end{itemize}
\end{block}
\end{frame}


%!!!!!!!!!!!!!!!!!!!!!!!!!!!!!!!!!!!!!!!!!
\begin{frame}[containsverbatim]{\ftitle}
%_________________________________________
\def\blocktitle{makefile}
\begin{block}{\blocktitle}
\begin{verbatim}
all: main
  #vide
  
main: lmath.o main.o
  gcc -o executable main.o lmath.o
  
lmath.o: lmath.c
  gcc -c -o lmath.o lmath.h
  
main.o: main.c lmath.h
  gcc -c -o main.o main.c

clean:
  rm -rf *.o
  
mrproper:
  rm -rf executable
 
\end{verbatim}
\end{block}
\end{frame}



%!!!!!!!!!!!!!!!!!!!!!!!!!!!!!!!!!!!!!!!!!
\def\ftitle{Encore plus}
\begin{frame}[containsverbatim]{\ftitle}
%_________________________________________
\def\blocktitle{Variables}
\begin{block}{\blocktitle}
\begin{verbatim}
#Compilateur
CC=gcc
#Option de compilation
CFLAGS=-Wall
#Option de linkage
LDFLAGS=-lm
#Nom de l'exécutable
EXEC=main

all: $(EXEC)
  #vide
  
main: lmath.o main.o
  $(CC) -o executable main.o lmath.o $(LDFLAGS)
  

 
\end{verbatim}%$
\end{block}
\end{frame}


%!!!!!!!!!!!!!!!!!!!!!!!!!!!!!!!!!!!!!!!!!
\def\ftitle{Introduction}
\begin{frame}[containsverbatim]{\ftitle}
%_________________________________________
\def\blocktitle{Lexical}
\begin{block}{\blocktitle}
\begin{verbatim}
lmath.o: lmath.c
  $(CC) -c -o lmath.o lmath.h $(CFLAGS)
  
main.o: main.c lmath.h
  $(CC) -c -o main.o main.c $(CFLAGS)

clean:
  rm -rf *.o
  
mrproper:
  rm -rf executable
\end{verbatim}
\end{block}
\end{frame}

%!!!!!!!!!!!!!!!!!!!!!!!!!!!!!!!!!!!!!!!!!
\begin{frame}[containsverbatim]{\ftitle}
%_________________________________________
\def\blocktitle{Variable propre a makefile}
\begin{block}{\blocktitle}
\begin{itemize}
\item \verb!$@! est le nom de la cible en cours
\item \verb!$@! est le nom du fichier sans extension
\item \verb!$<! est le nom de la première dépendance
\item \verb!$^! est le nom de toutes les dépendances
\item \verb!$?! est le nom des dépendances plus récentes que la cible
\end{itemize}%$
\end{block}
\end{frame}


%_________________________________________
\def\blocktitle{On peut écrire}
\begin{block}{\blocktitle}
\begin{verbatim}
main: lmath.o main.o
  $(CC) -o $@ $^ $(LDFLAGS)
  
\end{verbatim}%$
Va compiler avec gcc -o nom\_de\_la\_cible dépendances flags\_de\_compilation
\end{block}


%!!!!!!!!!!!!!!!!!!!!!!!!!!!!!!!!!!!!!!!!!
\begin{frame}[containsverbatim]{\ftitle}
%_________________________________________
\def\blocktitle{Vocabulaire}
\begin{block}{\blocktitle}
\begin{itemize}
\item Si on regarde notre makefile on verra qu'il y a des règles qui fonctionnent de la même manière
\begin{verbatim}
lmath.o: lmath.c
  $(CC) -c -o lmath.o lmath.h $(CFLAGS)
  
main.o: main.c lmath.h
  $(CC) -c -o main.o main.c $(CFLAGS)
  
\end{verbatim}
\item On peut généraliser
\end{itemize}
\end{block}
\end{frame}


%!!!!!!!!!!!!!!!!!!!!!!!!!!!!!!!!!!!!!!!!!
\def\ftitle{Introduction}
\begin{frame}[containsverbatim]{\ftitle}
%_________________________________________
\def\blocktitle{Règle globale}
\begin{block}{\blocktitle}
\begin{itemize}
\item \begin{verbatim}
%.o:%.c
  $(CC) -c -o $@ $< $(CFLAGS)
  
\end{verbatim}
\item Cette règle compile tout les fichiers .o de la même manière. (gcc -c -o nom\_de\_la\_cible nom\_du\_fichier.c flags)
\item Attention main.o ne dépend plus de lmath.h
\item \begin{verbatim}
main.o: lmath.h
\end{verbatim}
\end{itemize}

\end{block}
\end{frame}

%!!!!!!!!!!!!!!!!!!!!!!!!!!!!!!!!!!!!!!!!!
\begin{frame}[containsverbatim]{\ftitle}
%_________________________________________
\def\blocktitle{Petites substitutions}
\begin{block}{\blocktitle}
\begin{itemize}
\item \verb!SRC=main.c lmath.c! \verb!OBJ=$(SRC:.c=.o)!
\item \verb!src=$(wildcard *.C)!
\end{itemize}
\end{block}
\end{frame}

\end{document}
